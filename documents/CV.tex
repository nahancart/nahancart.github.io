% Resume 2.0 (2009-04-12)
% Copyright 2009 Zach Scrivena
% zachscrivena@gmail.com
% http://zs.freeshell.org/latexresume/
%
% This document is released under the
% Creative Commons Attribution 3.0 Unported License
% http://creativecommons.org/licenses/by/3.0/

%%%%%%%%%%%%%%
%% PREAMBLE %%
%%%%%%%%%%%%%%

\documentclass[letterpaper,10pt,oneside]{article}
%\documentclass[a4paper,12pt,oneside]{article}
\usepackage{ifthen}
\usepackage{longtable}
\usepackage{setspace}
\usepackage[left=.8in,right=.8in,top=.6in,bottom=0.6in,nohead,nofoot]{geometry}
\usepackage{marvosym}
\usepackage{lmodern}
\usepackage{multicol}
\usepackage{comment}

\usepackage[bookmarks,
pdftitle={Resume},
pdfauthor={John Doe},
pdfsubject={},
pdfcreator={},
pdfproducer={},
pdfkeywords={},
pdfpagemode={},
pdfstartview=FitH,
colorlinks=true,
linkcolor=black,   %Color for normal internal links.
anchorcolor=black, %Color for anchor text.
citecolor=green,   %Color for bibliographical citations in text.
filecolor=magenta, %Color for URLs which open local files.
menucolor=red,     %Color for Acrobat menu items.
urlcolor=black]{hyperref}

\newcommand{\longmonth}[1]{%
\ifthenelse{\equal{#1}{1}}{January}{}%
\ifthenelse{\equal{#1}{2}}{February}{}%
\ifthenelse{\equal{#1}{3}}{March}{}%
\ifthenelse{\equal{#1}{4}}{April}{}%
\ifthenelse{\equal{#1}{5}}{May}{}%
\ifthenelse{\equal{#1}{6}}{June}{}%
\ifthenelse{\equal{#1}{7}}{July}{}%
\ifthenelse{\equal{#1}{8}}{August}{}%
\ifthenelse{\equal{#1}{9}}{September}{}%
\ifthenelse{\equal{#1}{01}}{January}{}%
\ifthenelse{\equal{#1}{02}}{February}{}%
\ifthenelse{\equal{#1}{03}}{March}{}%
\ifthenelse{\equal{#1}{04}}{April}{}%
\ifthenelse{\equal{#1}{05}}{May}{}%
\ifthenelse{\equal{#1}{06}}{June}{}%
\ifthenelse{\equal{#1}{07}}{July}{}%
\ifthenelse{\equal{#1}{08}}{August}{}%
\ifthenelse{\equal{#1}{09}}{September}{}%
\ifthenelse{\equal{#1}{10}}{October}{}%
\ifthenelse{\equal{#1}{11}}{November}{}%
\ifthenelse{\equal{#1}{12}}{December}{}}

\newcommand{\shortmonth}[1]{%
\ifthenelse{\equal{#1}{1}}{Jan}{}%
\ifthenelse{\equal{#1}{2}}{Feb}{}%
\ifthenelse{\equal{#1}{3}}{Mar}{}%
\ifthenelse{\equal{#1}{4}}{Apr}{}%
\ifthenelse{\equal{#1}{5}}{May}{}%
\ifthenelse{\equal{#1}{6}}{Jun}{}%
\ifthenelse{\equal{#1}{7}}{Jul}{}%
\ifthenelse{\equal{#1}{8}}{Aug}{}%
\ifthenelse{\equal{#1}{9}}{Sep}{}%
\ifthenelse{\equal{#1}{01}}{Jan}{}%
\ifthenelse{\equal{#1}{02}}{Feb}{}%
\ifthenelse{\equal{#1}{03}}{Mar}{}%
\ifthenelse{\equal{#1}{04}}{Apr}{}%
\ifthenelse{\equal{#1}{05}}{May}{}%
\ifthenelse{\equal{#1}{06}}{Jun}{}%
\ifthenelse{\equal{#1}{07}}{Jul}{}%
\ifthenelse{\equal{#1}{08}}{Aug}{}%
\ifthenelse{\equal{#1}{09}}{Sep}{}%
\ifthenelse{\equal{#1}{10}}{Oct}{}%
\ifthenelse{\equal{#1}{11}}{Nov}{}%
\ifthenelse{\equal{#1}{12}}{Dec}{}}

% Style settings
\pagestyle{empty}
\setlength{\parindent}{0in}
\renewcommand{\thefootnote}{\fnsymbol{footnote}} % change footnote symbols

% Macros
\newenvironment{myresume}[2]{\pdfbookmark{#1}{#1}{\LARGE\begin{center}\textbf{{#1}}\end{center}}\vspace{1mm}\vspace{-3mm}\begin{longtable}{p{0.12\textwidth} p{0.83\textwidth}}~&#2}{\end{longtable}}
\renewcommand{\section}[2]{\vspace{-1.3em}\\\pdfbookmark{#1}{#2}{\singlespace #1}&}
\newcommand{\firstentry}[1]{\newline#1}
\newcommand{\entry}[1]{\vspace{-0.8em}\\~&#1}
\newcommand{\hide}[1]{}

% Datestamp format: \datestamp{YEAR}{MONTH}{DAY}
%\newcommand{\datestamp}[3]{\longmonth{#2} #3, #1}
%\newcommand{\datestamp}[3]{\shortmonth{#2} #3, #1}
\newcommand{\datestamp}[3]{\shortmonth{#2} #1}

%%%%%%%%%%%%%%%%
%% NAME BLOCK %%
%%%%%%%%%%%%%%%%

\begin{document}
{\fontfamily{lmss}\selectfont
\begin{myresume}{Nathan Hancart} 

\hspace{-2.5cm}{\small
Department of Economics, University College London      \hfill \href{mailto:nathan.hancart.16@ucl.ac.uk}{\textcolor{blue}{nathan.hancart.16@ucl.ac.uk}}

\hspace{-2.5cm}30 Gordon Street, London WC1H 0AX, UK \hfill \href{www.nathanhancart.com}{\textcolor{blue}{nathanhancart.com}}
}%

\\\hline

%%%%%%%%%%%%%%%
%% EDUCATION %%
%%%%%%%%%%%%%%%

\section{Current\newline Position}{Current Position}

\firstentry{
\textbf{PhD Candidate in Economics}, University College London\hfill 2018 -- present \newline
Expected Completion: 2023
}

\section{Fields}{Fields of Interest}

\firstentry{
Primary: Microeconomic Theory\newline Secondary: Information Economics, Behavioral Economics
}
\begin{comment}


\section{Placement Details}{Placement Details}

\firstentry{
Placement Director: Prof. Franck Portier\hfill
{\href{mailto:f.portier@ucl.ac.uk}{\textcolor{blue}{f.portier@ucl.ac.uk}}}
\newline
Graduate Coordinator: Daniella Harper\hfill
{\href{mailto:economics.jobmarket@ucl.ac.uk}{\textcolor{blue}{economics.jobmarket@ucl.ac.uk}}}%
}

\end{comment}
\vspace{0.3cm}
\section{Supervisors}{Supervisors}


\vspace{0.1cm}

%
\begin{minipage}[ht]{0.48\textwidth}
Prof. Ran Spiegler \\
\begin{comment}
UCL Department of Economics \\
Drayton House, 30 Gordon St \\
London, WC1H 0AX\\
{\href{mailto:r.spiegler@ucl.ac.uk}{\textcolor{blue}{r.spiegler@ucl.ac.uk}}}\\
\end{comment}
\end{minipage}%
\hspace{-3cm}\begin{minipage}[ht]{0.48\textwidth}\vspace{-0.1cm}
%\vspace{-0.4cm}
Prof. Vasiliki Skreta \\
\begin{comment}
UCL Department of Economics \\
Drayton House, 30 Gordon St \\
London, WC1H 0AX\\
{\href{mailto:v.skreta@ucl.ac.uk}{\textcolor{blue}{v.skreta@ucl.ac.uk}}}\\
\end{comment}
\end{minipage}

\begin{comment}
\begin{minipage}[ht]{0.48\textwidth}
Prof. Xxx Xxx\\
UCL Department of Economics \\
Drayton House, 30 Gordon St \\
London, WC1H 0AX\\
{\href{mailto:xxx@ucl.ac.uk}{\textcolor{blue}{xxx@ucl.ac.uk}}}\\
\end{minipage}
\end{comment}
%%%%%%%%%%%%%%%%%%%%%%%%%
%% Education %%
%%%%%%%%%%%%%%%%%%%%%%%%%

\section{Prior\newline Education}{Education}

\firstentry{
\textbf{MRes} in Economics, University College London\hfill 2018 
}
\firstentry{
\textbf{MSc} in Economics, University College London\hfill 2017 
}%
\entry{
\textbf{BSc} in Business Engineering, Université Libre de Bruxelles\hfill 2016 
}

%%%%%%%%%%%%%%%%%%
%% Job Market Paper %%
%%%%%%%%%%%%%%%%%%

\section{Job Market\newline Paper}{Job Market Paper}

\firstentry{\textbf{Optimal Menus of Tests}\vspace{0.1cm}\newline
I study the optimal design of menus of tests. Prior to taking a binary decision, accept or reject a privately informed agent, a decision-maker (DM) can perform one test from a restricted set. For example, the restriction can come from information processing or technological constraints. The DM wants to accept a subset of types whereas the agent always wants to be accepted. Instead of choosing the test himself, the DM let the agent choose a test from a menu. The choice itself then serves as an additional dimension for information revelation. I characterise when a menu is optimal and show that the DM does not benefit from committing to an action. Using these results, I show conditions under which the DM wants or does not want to include strictly less informative test in the menu. I also define an order on tests that characterises which tests are part of an optimal menu. Finally, I demonstrate in an application how the need to provide incentives for self-selection in a menu hinders the ability to learn from the test itself.  
}

%%%%%%%%%%%%%%%%%%
%% Other Research %%
%%%%%%%%%%%%%%%%%%

\section{Working\newline Paper}{Other Research}

\firstentry{\textbf{Managing the Expectations of Buyers with Reference-dependent Preferences}\newline
I consider a model of monopoly pricing where a risk-neutral firm makes an offer to a buyer with reference-dependent preferences. The reference point is the ex-ante probability of trade and the buyer exhibits an attachment effect: the higher his expectations to buy, the higher his willingness-to-pay. When the buyer’s valuation is private information, a unique equilibrium exists where the firm plays a mixed strategy and its profits are the same as in the reference-independent benchmark. The equilibrium always entails inefficiencies: even as the firm’s information converges to complete information, it mixes on a non-vanishing support and the probability of no trade is greater than zero. Finally, I show that when the firm can obtain costless signals on the buyer’s valuation,
it can do strictly better than in the reference-independent benchmark by leveraging the uncertainty generated by a noisy learning strategy. However, this advantage vanishes as the attachment effect grows large.
}


%%%%%%%%%%%%%%%%%%%%%%%%%%%%%%
%% Teaching %%
%%%%%%%%%%%%%%%%%%%%%%%%%%%%%%

\section{Teaching}{teaching}

\firstentry{\textbf{Microeconomics} (MRes, UCL) \hfill \datestamp{2018}{}{} -- present
}%
\entry{\textbf{Advanced Microeconomic Theory} (MSc, UCL) \hfill \datestamp{2018}{}{} -- \datestamp{2020}{}{}
}%
\entry{\textbf{Economics of Information} (BSc, UCL) \hfill \datestamp{2017}{}{} -- \datestamp{2019}{}{}
}


%%%%%%%%%%%%%%%%%%
%% AWARDS %%
%%%%%%%%%%%%%%%%%%

%\section{Honors \&\newline Awards}{honors}

%\firstentry{Award for best Teaching Assistant on an MRes module, UCL\hfill \datestamp{2022}{06}{}
%}

%%%%%%%%%%%%%%%%%%%%%%%%%%%%%%
%% Referee Service %%
%%%%%%%%%%%%%%%%%%%%%%%%%%%%%%

\section{Referee\newline Service}{Referee Service}

\firstentry{Theoretical Economics
}%

%%%%%%%%%%%%%%%%%%%%%%%%%%%%%%
%% Work %%
%%%%%%%%%%%%%%%%%%%%%%%%%%%%%%

\section{Professional Experience}{experience}

\firstentry{Research assistant for Prof. Ran Spiegler \hfill \datestamp{2018}{04}{} -- \datestamp{2022}{}{}\newline
Research assistant for Prof. Vasiliki Skreta \hfill \datestamp{2019}{}{} -- 2022
}%

\entry{Student Representative for PhD students at the Economics Department \hfill \datestamp{2017}{}{} -- present
}%

%%%%%%%%%%%%%%%%%%%%%%%%%%%%%%
%% Presentation %%
%%%%%%%%%%%%%%%%%%%%%%%%%%%%%%
\section{Presentations}{presentations}

\firstentry{Brown-bag theory seminar UCL (2022, 2021), World Congress Game Theory Society (Budapest, 2021), Applied Theory Workshop (Toulouse School of Economics, 2020)}
%\firstentry{\hfill \datestamp{}{}{}
%}

%%%%%%%%%%%%%%%
%% LANGUAGES %%
%%%%%%%%%%%%%%%

\section{Languages}{languages}

\firstentry{French (Native), English (Fluent), Dutch (Basic), Hebrew (Basic)}

%%%%%%%%%%%%
%% SKILLS %%
%%%%%%%%%%%%

\section{Software Skills}{skills}

\firstentry{Mathematica, Matlab}

\end{myresume}}
\end{document}