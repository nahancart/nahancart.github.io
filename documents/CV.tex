% Resume 2.0 (2009-04-12)
% Copyright 2009 Zach Scrivena
% zachscrivena@gmail.com
% http://zs.freeshell.org/latexresume/
%
% This document is released under the
% Creative Commons Attribution 3.0 Unported License
% http://creativecommons.org/licenses/by/3.0/

%%%%%%%%%%%%%%
%% PREAMBLE %%
%%%%%%%%%%%%%%

\documentclass[letterpaper,10pt,oneside]{article}
%\documentclass[a4paper,12pt,oneside]{article}
\usepackage{ifthen}
\usepackage{longtable}
\usepackage{setspace}
\usepackage[left=.8in,right=.8in,top=.6in,bottom=0.6in,nohead,nofoot]{geometry}
\usepackage{marvosym}
\usepackage{lmodern}
\usepackage{multicol}
\usepackage{comment}

\usepackage[bookmarks,
pdftitle={Resume},
pdfauthor={John Doe},
pdfsubject={},
pdfcreator={},
pdfproducer={},
pdfkeywords={},
pdfpagemode={},
pdfstartview=FitH,
colorlinks=true,
linkcolor=black,   %Color for normal internal links.
anchorcolor=black, %Color for anchor text.
citecolor=green,   %Color for bibliographical citations in text.
filecolor=magenta, %Color for URLs which open local files.
menucolor=red,     %Color for Acrobat menu items.
urlcolor=black]{hyperref}

\newcommand{\longmonth}[1]{%
\ifthenelse{\equal{#1}{1}}{January}{}%
\ifthenelse{\equal{#1}{2}}{February}{}%
\ifthenelse{\equal{#1}{3}}{March}{}%
\ifthenelse{\equal{#1}{4}}{April}{}%
\ifthenelse{\equal{#1}{5}}{May}{}%
\ifthenelse{\equal{#1}{6}}{June}{}%
\ifthenelse{\equal{#1}{7}}{July}{}%
\ifthenelse{\equal{#1}{8}}{August}{}%
\ifthenelse{\equal{#1}{9}}{September}{}%
\ifthenelse{\equal{#1}{01}}{January}{}%
\ifthenelse{\equal{#1}{02}}{February}{}%
\ifthenelse{\equal{#1}{03}}{March}{}%
\ifthenelse{\equal{#1}{04}}{April}{}%
\ifthenelse{\equal{#1}{05}}{May}{}%
\ifthenelse{\equal{#1}{06}}{June}{}%
\ifthenelse{\equal{#1}{07}}{July}{}%
\ifthenelse{\equal{#1}{08}}{August}{}%
\ifthenelse{\equal{#1}{09}}{September}{}%
\ifthenelse{\equal{#1}{10}}{October}{}%
\ifthenelse{\equal{#1}{11}}{November}{}%
\ifthenelse{\equal{#1}{12}}{December}{}}

\newcommand{\shortmonth}[1]{%
\ifthenelse{\equal{#1}{1}}{Jan}{}%
\ifthenelse{\equal{#1}{2}}{Feb}{}%
\ifthenelse{\equal{#1}{3}}{Mar}{}%
\ifthenelse{\equal{#1}{4}}{Apr}{}%
\ifthenelse{\equal{#1}{5}}{May}{}%
\ifthenelse{\equal{#1}{6}}{Jun}{}%
\ifthenelse{\equal{#1}{7}}{Jul}{}%
\ifthenelse{\equal{#1}{8}}{Aug}{}%
\ifthenelse{\equal{#1}{9}}{Sep}{}%
\ifthenelse{\equal{#1}{01}}{Jan}{}%
\ifthenelse{\equal{#1}{02}}{Feb}{}%
\ifthenelse{\equal{#1}{03}}{Mar}{}%
\ifthenelse{\equal{#1}{04}}{Apr}{}%
\ifthenelse{\equal{#1}{05}}{May}{}%
\ifthenelse{\equal{#1}{06}}{Jun}{}%
\ifthenelse{\equal{#1}{07}}{Jul}{}%
\ifthenelse{\equal{#1}{08}}{Aug}{}%
\ifthenelse{\equal{#1}{09}}{Sep}{}%
\ifthenelse{\equal{#1}{10}}{Oct}{}%
\ifthenelse{\equal{#1}{11}}{Nov}{}%
\ifthenelse{\equal{#1}{12}}{Dec}{}}

% Style settings
\pagestyle{empty}
\setlength{\parindent}{0in}
\renewcommand{\thefootnote}{\fnsymbol{footnote}} % change footnote symbols

% Macros
\newenvironment{myresume}[2]{\pdfbookmark{#1}{#1}{\LARGE\begin{center}\textbf{{#1}}\end{center}}\vspace{1mm}\vspace{-3mm}\begin{longtable}{p{0.12\textwidth} p{0.83\textwidth}}~&#2}{\end{longtable}}
\renewcommand{\section}[2]{\vspace{-1.3em}\\\pdfbookmark{#1}{#2}{\singlespace #1}&}
\newcommand{\firstentry}[1]{\newline#1}
\newcommand{\entry}[1]{\vspace{-0.8em}\\~&#1}
\newcommand{\hide}[1]{}

% Datestamp format: \datestamp{YEAR}{MONTH}{DAY}
%\newcommand{\datestamp}[3]{\longmonth{#2} #3, #1}
%\newcommand{\datestamp}[3]{\shortmonth{#2} #3, #1}
\newcommand{\datestamp}[3]{\shortmonth{#2} #1}

%%%%%%%%%%%%%%%%
%% NAME BLOCK %%
%%%%%%%%%%%%%%%%

\begin{document}
{\fontfamily{lmss}\selectfont
\begin{myresume}{Nathan Hancart} 

\hspace{-2.5cm}{\small
Department of Economics, University College London      \hfill \href{mailto:nathan.hancart.16@ucl.ac.uk}{\textcolor{blue}{nathan.hancart.16@ucl.ac.uk}}

\hspace{-2.5cm}30 Gordon Street, London WC1H 0AX, UK \hfill \href{www.nathanhancart.com}{\textcolor{blue}{nathanhancart.com}}
}%

\\\hline

%%%%%%%%%%%%%%%
%% EDUCATION %%
%%%%%%%%%%%%%%%

\section{Current\newline Position}{Current Position}

\firstentry{
\textbf{PhD Candidate in Economics}, University College London\hfill 2018 -- present \newline
Expected Completion: 2023
}

\section{Fields}{Fields of Interest}

\firstentry{
Primary: Microeconomic Theory\newline Secondary: Information Economics, Behavioural Economics
}
\begin{comment}


\section{Placement Details}{Placement Details}

\firstentry{
Placement Director: Prof. Franck Portier\hfill
{\href{mailto:f.portier@ucl.ac.uk}{\textcolor{blue}{f.portier@ucl.ac.uk}}}
\newline
Graduate Coordinator: Daniella Harper\hfill
{\href{mailto:economics.jobmarket@ucl.ac.uk}{\textcolor{blue}{economics.jobmarket@ucl.ac.uk}}}%
}

\end{comment}
\vspace{0.3cm}
\section{Supervisors}{Supervisors}


\vspace{0.1cm}

%
\begin{minipage}[ht]{0.48\textwidth}
Prof. Ran Spiegler \\
\begin{comment}
UCL Department of Economics \\
Drayton House, 30 Gordon St \\
London, WC1H 0AX\\
{\href{mailto:r.spiegler@ucl.ac.uk}{\textcolor{blue}{r.spiegler@ucl.ac.uk}}}\\
\end{comment}
\end{minipage}%
\hspace{-3cm}\begin{minipage}[ht]{0.48\textwidth}\vspace{-0.1cm}
%\vspace{-0.4cm}
Prof. Vasiliki Skreta \\
\begin{comment}
UCL Department of Economics \\
Drayton House, 30 Gordon St \\
London, WC1H 0AX\\
{\href{mailto:v.skreta@ucl.ac.uk}{\textcolor{blue}{v.skreta@ucl.ac.uk}}}\\
\end{comment}
\end{minipage}

\begin{comment}
\begin{minipage}[ht]{0.48\textwidth}
Prof. Xxx Xxx\\
UCL Department of Economics \\
Drayton House, 30 Gordon St \\
London, WC1H 0AX\\
{\href{mailto:xxx@ucl.ac.uk}{\textcolor{blue}{xxx@ucl.ac.uk}}}\\
\end{minipage}
\end{comment}
%%%%%%%%%%%%%%%%%%%%%%%%%
%% Education %%
%%%%%%%%%%%%%%%%%%%%%%%%%

\section{Prior\newline Education}{Education}

\firstentry{
\textbf{MRes} in Economics, University College London\hfill 2018 
}
\firstentry{
\textbf{MSc} in Economics, University College London\hfill 2017 
}%
\entry{
\textbf{BSc} in Business Engineering, Université Libre de Bruxelles\hfill 2016 
}

%%%%%%%%%%%%%%%%%%
%% Job Market Paper %%
%%%%%%%%%%%%%%%%%%

\section{Job Market\newline Paper}{Job Market Paper}

\firstentry{\textbf{Designing the Optimal Menu of Tests}\vspace{0.1cm}\newline
A decision-maker (DM) must accept or reject a privately informed agent. Prior to taking his decision, he can perform one test from a restricted set of feasible tests. The agent always wants to be accepted while the DM wants to accept a subset of types. The DM designs a menu of test and lets the agent choose a test from the menu. The choice itself then serves as an additional source of information. I provide a characterisation of the DM-optimal menu for arbitrary type structures and feasible tests. I also show that the DM does not benefit from committing ex-ante to a strategy. Using these results, I characterise the optimal menu when the set of feasible tests contains a most informative test. I give conditions on the DM's preferences under which a strictly less informative test is included in the menu. I also characterise the optimal menu when types are multidimensional or when tests vary in their difficulty. 
}

%%%%%%%%%%%%%%%%%%
%% Other Research %%
%%%%%%%%%%%%%%%%%%

\section{Working\newline Paper}{Other Research}

\firstentry{\textbf{Managing the Expectations of Buyers with Reference-dependent Preferences} \newline\textit{R\&R at Journal of Economic Theory}\newline
I consider a model of monopoly pricing where a risk-neutral firm makes an offer to a buyer with reference-dependent preferences. The reference point is the ex-ante probability of trade and the buyer exhibits an attachment effect: the higher his expectations to buy, the higher his willingness-to-pay. When the buyer’s valuation is private information, a unique equilibrium exists where the firm plays a mixed strategy and its profits are the same as in the reference-independent benchmark. The equilibrium always entails inefficiencies: even as the firm’s information converges to complete information, it mixes on a non-vanishing support and the probability of no trade is greater than zero. Finally, I show that when the firm can obtain costless signals on the buyer’s valuation,
it can do strictly better than in the reference-independent benchmark by leveraging the uncertainty generated by a noisy learning strategy. However, this advantage vanishes as the attachment effect grows large.
}

\entry{\textbf{The (No) Value of Commitment} \newline
 I provide a sufficient condition under which a principal does not benefit from commitment in economic situations that can be described by a constrained maximisation problem. I show that commitment has no value when the \textit{marginal} contribution of the constraints is null. I then apply this condition in a mechanism design setting. A designer does not benefit from being able to contract over actions when his preferences are partially aligned with the agent's. Verifying the condition does not necessitate verifying explicitly that the strategy under commitment is a best-response to the information revealed in the economic problem. }


%%%%%%%%%%%%%%%%%%%%%%%%%%%%%%
%% Teaching %%
%%%%%%%%%%%%%%%%%%%%%%%%%%%%%%

\section{Teaching}{teaching}

\firstentry{\textbf{Microeconomics} (MRes, UCL) \hfill \datestamp{2018}{}{} --  \datestamp{2022}{}{}
}%
\entry{\textbf{Econometrics} (BSc, University of London - External Program) \hfill \datestamp{2022}{}{} -- \datestamp{2023}{}{}
}
\entry{\textbf{Advanced Microeconomic Theory} (MSc, UCL) \hfill \datestamp{2018}{}{} -- \datestamp{2020}{}{}
}%
\entry{\textbf{Economics of Information} (BSc, UCL) \hfill \datestamp{2017}{}{} -- \datestamp{2019}{}{}
}%
\entry{\textbf{Econometrics} (BSc, University of London - External Program) \hfill \datestamp{2022}{}{} -- \datestamp{2023}{}{}
}



%%%%%%%%%%%%%%%%%%
%% AWARDS %%
%%%%%%%%%%%%%%%%%%

\section{Honors \&\newline Awards}{honors}

\firstentry{Award for best Teaching Assistant on an MRes module, UCL\hfill \datestamp{2022}{06}{}\newline%
Nominated for Inspiring Teaching Delivery, Student Choice Award, UCL\hfill \datestamp{2020}{06}{}
}%

%%%%%%%%%%%%%%%%%%%%%%%%%%%%%%
%% Referee Service %%
%%%%%%%%%%%%%%%%%%%%%%%%%%%%%%

\section{Referee\newline Service}{Referee Service}

\firstentry{Theoretical Economics
}%

%%%%%%%%%%%%%%%%%%%%%%%%%%%%%%
%% Work %%
%%%%%%%%%%%%%%%%%%%%%%%%%%%%%%

\section{Professional Experience}{experience}

\firstentry{Research assistant for Prof. Ran Spiegler \hfill \datestamp{2018}{04}{} -- \datestamp{2022}{}{}\newline
Research assistant for Prof. Vasiliki Skreta \hfill \datestamp{2019}{}{} -- \datestamp{2022}{}{}
}%

\entry{Student Representative for PhD students at the Economics Department \hfill \datestamp{2017}{}{} -- \datestamp{2022}{}{}
}%

%%%%%%%%%%%%%%%%%%%%%%%%%%%%%%
%% Presentation %%
%%%%%%%%%%%%%%%%%%%%%%%%%%%%%%
\section{Presentations}{presentations}

\firstentry{Theory Seminar (Center for the Study of Rationality, HU Jerusalem, Scheduled 2022), Economic Theory Workshop (Tel-Aviv University, Scheduled 2022), Internal Theory Workshop (Penn State, Scheduled 2022), Asian School in Economic Theory (National University of Singapore/Econometric Society, 2022), International Conference on Game Theory (Stony Brook, 2022), Brown-bag theory seminar (UCL, 2022/2021), World Congress Game Theory Society (Budapest, 2021), Applied Theory Workshop (Toulouse School of Economics, 2020)}
%\firstentry{\hfill \datestamp{}{}{}
%}

%%%%%%%%%%%%%%%
%% LANGUAGES %%
%%%%%%%%%%%%%%%

\section{Languages}{languages}

\firstentry{French (Native), English (Fluent), Dutch (Basic), Hebrew (Basic)}

%%%%%%%%%%%%
%% SKILLS %%
%%%%%%%%%%%%

\section{Software Skills}{skills}

\firstentry{Mathematica, Matlab}

\end{myresume}}
\end{document}